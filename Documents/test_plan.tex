
\documentclass[11pt]{article} % use larger type; default would be 10pt

\usepackage[utf8]{inputenc} % set input encoding (not needed with XeLaTeX)


\usepackage{geometry} % to change the page dimensions
\geometry{a4paper} % or letterpaper (US) or a5paper or....

\usepackage{graphicx} % support the \includegraphics command and options

\usepackage{booktabs} % for much better looking tables
\usepackage{array} % for better arrays (eg matrices) in maths
\usepackage{paralist} % very flexible & customisable lists (eg. enumerate/itemize, etc.)
\usepackage{verbatim} % adds environment for commenting out blocks of text & for better verbatim
\usepackage{subfig} % make it possible to include more than one captioned figure/table in a single float
% These packages are all incorporated in the memoir class to one degree or another...

%%% HEADERS & FOOTERS
\usepackage{fancyhdr} % This should be set AFTER setting up the page geometry
\pagestyle{fancy} % options: empty , plain , fancy
\renewcommand{\headrulewidth}{0pt} % customise the layout...
\lhead{}\chead{}\rhead{}
\lfoot{}\cfoot{\thepage}\rfoot{}

%%% SECTION TITLE APPEARANCE
\usepackage{sectsty}
\allsectionsfont{\sffamily\mdseries\upshape} % (See the fntguide.pdf for font help)
% (This matches ConTeXt defaults)

%%% ToC (table of contents) APPEARANCE
\usepackage[nottoc,notlof,notlot]{tocbibind} % Put the bibliography in the ToC
\usepackage[titles,subfigure]{tocloft} % Alter the style of the Table of Contents


\usepackage{float} % taulukoiden oikein kohdistus

\title{Haskell - WebAssembly -kääntäjän testit}
\author{-}
%\date{} % Activate to display a given date or no date (if empty),
         % otherwise the current date is printed 

\begin{document}
\maketitle


\section{Testitapaukset: yksikkötestit (Hassembler.tests)}

Näissä taulukoissa on kuvattu yksikkötestit, jotka sijaitsevat Hassembler.tests-projektissa.

\begin{table}[!htbp] %'!htbp places the table correctly
\caption{Testitapaukset: yksikkötestit (HassemblerTests)}
\begin{tabular}{|p{3cm}|p{2,5cm}|p{3cm}|p{1,5cm}|p{3cm}|p{1,5cm}|}
\hline
\textbf{Nimi} & \textbf{Mitä testaa} & \textbf{Lähdekielinen sisältö} & \textbf{Syöte}& \textbf{Odotettu tulos (tulkki)}  & \textbf{Vaihe} \\ \hline
 Arith()             & perus\-aritmetiikka, muuttujat (binding)         & f = 1+2*3  & f  & f = 7                                  &  2/3              \\ \hline
 Arith2()             & perus\-aritmetiikka         & f = (10-1)/3  & f  & f = 3                                  &  2/3              \\ \hline
 ParenTest()              & sulkulausekkeet    &  f = ((2+(2*30))/(2*1))             &  f     & f = 31     &  2  \\ \hline             

 GetFuns()             & funktioiden listaus   & a=1+1+1*3\textbackslash r \textbackslash n f=2+4 \textbackslash r \textbackslash n &  &  ``a f "         &   3/5          \\ \hline
 BoolTest()    & totuusarvot ja funktion arvo  & f = 2 == 2   & f & f = True         & 4    \\ \hline
 IfThenElse()    & ehtolauseet                & f = if 1 $<$ 3 then 4+4 else 2 + 3   & f & f = 8         & 4    \\ \hline
 FunWithRef()  & funktiokutsut      &   a=3 \textbackslash r \textbackslash n f=a+9          &  f         & f = 12   & 3/5                                        \\ \hline
 FunCallWith\-Parameters()   &  funktiokutsut (parametrien kanssa)  &  f x y = x + y \textbackslash n   y=4 + f 2 3 &      f & f = 9        &  5/6             \\ \hline
 FunCallWith\-NaNPara\-meters()   &  funktiokutsut (ei-numero-parametrien kanssa)  &  f a = if a then True else False \textbackslash r \textbackslash n g = True &      f & f True = True        &  5/6             \\ \hline
 Recursion()   & rekursiivinen kutsu      &  f x y end = if y $<$ end then f y (x+y) end else y"    & f 1 1 10  & f = 13         
& 6         \\ \hline
FunWithSpaces\-AroundNewline()       &                      &                       &       &     &                                                   \\ \hline             
CommentTest()       &  kommentit       &   -- tämä on kommentti \textbackslash r \textbackslash n f = 2+2                    &  f     & f = 4    &   2              \\ \hline             
\end{tabular}
\end{table}


\begin{table}[!htbp] %'!htbp places the table correctly
\caption{Testitapaukset: yksikkötestit virhetilanteille (HassemblerTests)}
\begin{tabular}{|p{3cm}|p{2,5cm}|p{3cm}|p{1,5cm}|p{3cm}|p{1,5cm}|}
\hline
\textbf{Nimi} & \textbf{Mitä testaa} & \textbf{Lähdekielinen sisältö} & \textbf{Syöte}  & \textbf{Odotettu tulos (tulkki)}  & \textbf{Vaihe} \\ \hline
 FuncNotDefined()  & funktiota ei määritelty   &  g = 1 + 3       &  f       &  ``Function not found: f''   &  3            \\ \hline
 SyntaxErrorTest()     & syntaksivirhe   &  g = 1 ++ 2    &           &   (throws Exception)   &    1     \\ \hline
NegativeInt()              & negatiiviset kokonaisluvut (ei kuulu kieleen)   &  f = -2-2   & &   (throws Exception)                                     &  2              \\ \hline       
MultFunTest()      & samannimistä funktiota ei saa määrittää kahdesti   & f = 1 + 2 \textbackslash r \textbackslash n f = 4   &        &  (throws Exception)    &    2                                               \\ \hline       
%             &                      &                       &           &                      &                            \\ \hline          
\end{tabular}
\end{table}

\pagebreak

\section{Testitapaukset: tiedostot (.hs)}

Tässä taulukossa on kuvattu testiohjelmat (.hs-päätteiset tiedostot), jotka sijaitsevat tests-kansiossa.

\begin{table}[!htbp] %'!htbp places the table correctly
\caption{Testitapaukset: tiedostot (.hs)}
\begin{tabular}{|p{2,5cm}|p{2,5cm}|p{2,5cm}|p{2cm}|p{3cm}|p{1,5cm}|}
\hline
\textbf{Nimi} & \textbf{Mitä testaa} & \textbf{Lähdekielinen sisältö} & \textbf{Odotettu tulos (tulkki)} & \textbf{Odotettu tulos (kääntäjä)} & \textbf{Vaihe} \\ \hline
test\_module  &  minimaalinen ohjelma  & module A where &  ``Compiled 1 module''    &    (module)    & 8   \\ \hline
 test\_arithm     & perus\-aritmetiikka         &  &                                  &                                   &   8       \\ \hline
%              &                      &                       &           &                                   &                \\ \hline       
%              &                      &                       &           &                                   &                \\ \hline       


\end{tabular}
\end{table}

\end{document}
